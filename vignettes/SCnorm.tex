%\VignetteIndexEntry{SCnorm Vignette}
%\VignettePackage{SCnorm}
%\VignetteEngine{knitr::knitr}

\documentclass{article}\usepackage[]{graphicx}\usepackage[usenames,dvipsnames]{color}
% maxwidth is the original width if it is less than linewidth
% otherwise use linewidth (to make sure the graphics do not exceed the margin)
\makeatletter
\def\maxwidth{ %
  \ifdim\Gin@nat@width>\linewidth
    \linewidth
  \else
    \Gin@nat@width
  \fi
}
\makeatother

\definecolor{fgcolor}{rgb}{0.251, 0.251, 0.251}
\newcommand{\hlnum}[1]{\textcolor[rgb]{0.816,0.125,0.439}{#1}}%
\newcommand{\hlstr}[1]{\textcolor[rgb]{0.251,0.627,0.251}{#1}}%
\newcommand{\hlcom}[1]{\textcolor[rgb]{0.502,0.502,0.502}{\textit{#1}}}%
\newcommand{\hlopt}[1]{\textcolor[rgb]{0,0,0}{#1}}%
\newcommand{\hlstd}[1]{\textcolor[rgb]{0.251,0.251,0.251}{#1}}%
\newcommand{\hlkwa}[1]{\textcolor[rgb]{0.125,0.125,0.941}{#1}}%
\newcommand{\hlkwb}[1]{\textcolor[rgb]{0,0,0}{#1}}%
\newcommand{\hlkwc}[1]{\textcolor[rgb]{0.251,0.251,0.251}{#1}}%
\newcommand{\hlkwd}[1]{\textcolor[rgb]{0.878,0.439,0.125}{#1}}%
\let\hlipl\hlkwb

\newenvironment{knitrout}{}{} % an empty environment to be redefined in TeX
\usepackage{alltt}
\usepackage{graphicx, graphics, epsfig,setspace,amsmath, amsthm}
\usepackage{natbib}
\usepackage{moreverb}

\RequirePackage[]{/Library/Frameworks/R.framework/Versions/4.0/Resources/library/BiocStyle/resources/tex/Bioconductor}
\AtBeginDocument{\bibliographystyle{/Library/Frameworks/R.framework/Versions/4.0/Resources/library/BiocStyle/resources/tex/unsrturl}}


\IfFileExists{upquote.sty}{\usepackage{upquote}}{}
\begin{document}

\title{SCnorm: robust normalization of
    single-cell RNA-seq data}
\author{Rhonda Bacher and Christina Kendziorski}
\maketitle
\tableofcontents
\setcounter{tocdepth}{2}

\section{Introduction}
\label{sec:intro}
Normalization is an important first step in analyzing RNA-seq expression data to allow for accurate 
% comparisons of a gene's expression across samples. Typically, normalization methods estimate a scale factor per sample to 
% adjust for differences in the amount of sequencing each sample receives. This is because increases in sequencing 
% typically lead to proportional increases in gene counts. However, in scRNA-seq data 
% sequencing depth does not affect gene counts equally. SCnorm (as detailed in Bacher* and Chu* {\it et al.}, {{2017}}) 
% is a normalization approach we developed for single-cell RNA-seq data (scRNA-seq). SCnorm groups genes based on their count-depth 
% relationship and within each group applies a quantile regression to 
% estimate scaling factors to remove the effect of sequencing depth from
% the counts. 
% 
% If you use SCnorm in published research, please cite:
% 
% Bacher R, Chu LF, Leng N, Gasch AP, Thomson J, Stewart R, Newton M,  
% Kendziorski C. SCnorm: robust normalization of single-cell RNA-seq data.
% Nature Methods. 14 (6), 584-586
% 
% (http://www.nature.com/nmeth/journal/v14/n6/full/nmeth.4263.html)
% 
% If after reading through this vignette and the FAQ section you have questions or problems using SCnorm,
% please post them to 
% https://support.bioconductor.org
% and tag "SCnorm". This will notify the package maintainers and benefit other users.
% 
% \section{Run SCnorm}
% \label{sec:quickstart}
% Before analysis can proceed, the SCnorm package must be installed.
% 
% The easiest way to install SCnorm is through Bioconductor
% (https://bioconductor.org/packages/SCnorm):
% <<eval=FALSE, echo=TRUE>>=
% if (!requireNamespace("BiocManager", quietly=TRUE))
%     install.packages("BiocManager")
% BiocManager::install("SCnorm")
% @
% 
% There are two additional ways to download SCnorm.
% 
% The first option is most useful if using a previous version of SCnorm 
% (which can be found at https://github.com/rhondabacher/SCnorm/releases).
% 
% The second option is to download the most recent development version of SCnorm. This version may be used with versions of R below 3.4. This version
% includes use of summarizedExperiment, but parallel computation is maintained using the parallel package.
% 
% 
% <<eval=FALSE, echo=TRUE>>=
% install.packages("SCnorm_x.x.x.tar.gz", repos=NULL, type="source")
% #OR
% library(devtools)
% devtools::install_github("rhondabacher/SCnorm", ref="devel")
% @
% 
% After successful installation, the package must be loaded into the working
% space:
% 
% <<eval=TRUE, echo=TRUE,warning=FALSE>>=
% library(SCnorm)
% @
% 
% \subsection{Required inputs}
% 
% {\bf \Robject{Data}}:  Input to SCnorm may be either of class SummarizedExperiment or a matrix.
% The expression matrix should be a $G-by-S$ matrix containing the
% expression values for each gene and each cell,
% where $G$ is the number of genes and $S$ is the number of cells/samples. 
% Gene names should be assigned as rownames and not a column in the matrix.
% Estimates of gene expression are typically obtained using RSEM, HTSeq,
% Cufflinks, Salmon or similar approaches. 
% 
% \noindent The object \Robject{ExampleSimSCData} is a simulated single-cell data matrix containing
% 5,000 rows of genes and 90 columns of cells.
% 
% <<eval=TRUE>>=
% data(ExampleSimSCData)
% ExampleSimSCData[1:5,1:5]
% str(ExampleSimSCData)
% @
% 
% SCnorm now implements the\Rclass{SingleCellExperiment} class. The input data should be formatted as following, where ExamplesSimSCData is the expression matrix:
% <<eval=TRUE>>=
% ExampleSimSCData <- SingleCellExperiment::SingleCellExperiment(assays = list('counts' = ExampleSimSCData))
% @
% 
% If the data was originally formatted as a \Rclass{SummarizedExperiment}, then it can be coerced using:
% <<eval=FALSE>>=
% ExampleSimSCData <- as(ExampleSimSCData, "SingleCellExperiment")
% @
% 
% 
% {\bf Conditions}: The object \Robject{Conditions} should be a vector of length $S$
% indicating which condition/group each cell belongs to. The order of this vector
% MUST match the order of the columns in the \Robject{Data} matrix.
% 
% <<eval=TRUE>>=
% Conditions = rep(c(1), each= 90)
% head(Conditions)
% @
% 
% 
% 
% \subsection{SCnorm: Check count-depth relationship}
% \label{sec:checkData}
% Before normalizing using SCnorm it is advised to check the relationship between expression counts and sequencing
% depth (referred to as the count-depth relationship) for your data.
% If all genes have a similar relationship then
% a global normalization strategy such as median-by-ratio in the DESeq package or TMM in edgeR will also be
% adequate. However, we find that when the count-depth relationship
% varies among genes using global scaling
% strategies leads to poor normalization. In these cases we strongly recommend proceeding with the
% normalization provided by SCnorm.
% 
% The \Rfunction{plotCountDepth} function will estimate the count-depth relationship for all
% genes. The genes are first divided into ten equally sized groups based on their non-zero median expression.
% The function will generate a plot of the distribution of slopes for each group. We also recommend checking a variety of
% filter options in case you find that only genes expressed in very few cells or
% very low expressors are the main concern.
% 
% The function \Rfunction{plotCountDepth} produces a plot as output which may be wrapped around
% a call to pdf() or png() to save it for future reference. The function also returns
% the information used for plotting which may be ultilized if a user would like to generate a more
% customized figure. The returned object is a list with length equal to the number of conditions and each element of the
% list is a mtrix containing each genes's assigned group (based on non-zero median expression) and its' estimated
% count-depth relationship (Slope).
% 
% 
% <<eval=TRUE>>=
% pdf("check_exampleData_count-depth_evaluation.pdf", height=5, width=7)
% countDeptEst <- plotCountDepth(Data = ExampleSimSCData, Conditions = Conditions,
%                  FilterCellProportion = .1, NCores=3)
% dev.off()
% str(countDeptEst)
% head(countDeptEst[[1]])
% @
% 
% Since gene expression increases proportionally with sequencing depth, we expect to find the estimated count-depth relationships near 1 for all genes.
% This is typically true for bulk RNA-seq datasets. However, it does not hold in most single-cell RNA-seq datasets.
% In this example data, the relationship is quite variable across genes.
% 
% \begin{figure}[h!]
% \centering
% \includegraphics[width=.4\textwidth]{check_exampleData_count-depth_evaluation.pdf}
% \caption{Evaluation of count-depth relationship in un-normalized data.}
% \end{figure}
% 
% 
% After normalization the count-depth relationship can be estimated on the normalized data, where slopes near zero indicate successful normalization.
% We can use the function \Rfunction{plotCountDepth} to evaluate normalized data:
% 
% <<eval=TRUE, fig.height=5, fig.width=7, fig.align='center', out.width='0.4\\textwidth'>>=
% ExampleSimSCData = SingleCellExperiment::counts(ExampleSimSCData)
% # Total Count normalization, Counts Per Million, CPM.
% ExampleSimSCData.CPM <- t((t(ExampleSimSCData) / colSums(ExampleSimSCData)) *
%                         mean(colSums(ExampleSimSCData)))
% 
% 
% countDeptEst.CPM <- plotCountDepth(Data = ExampleSimSCData,
%                                   NormalizedData = ExampleSimSCData.CPM,
%                                    Conditions = Conditions,
%                                    FilterCellProportion = .1, NCores=3)
% 
% str(countDeptEst.CPM)
% head(countDeptEst.CPM[[1]])
% @
% 
% If normalization is succesful the estimated count-depth relationship should be near zero for all genes. Here the highly expressed genes appear
% well normalized, while moderate and low expressors have been over-normalized and have negative slopes.
% 
% 
% 
% \subsection{SCnorm: Normalization}
% \label{sec:Normalization}
% SCnorm will normalize across cells to remove the effect of sequencing
% depth on the counts. The function \Rfunction{SCnorm} returns: the normalized expression counts, a list of
% genes which were not considered in the normalization due to filter options, and optionally a
% matrix of scale factors (if \Rcode{reportSF = TRUE}).
% 
% The default filter for SCnorm only considers genes having at least 10 non-zero expression values.
% The user may wish to adjust the filter and may do so by changing the value of FilterCellNum. The
% user may also wish to not normalize very low expressed genes. The parameter \Robject{FilterExpression} can be used
% to exclude genes which have non-zero medians lower than a threshold specified by \Robject{FilterExpression}
% (\Robject{FilterExpression}=0 is default).
% 
% SCnorm starts at $K = 1$, which normalizes the data assuming all genes should be normalized in one group.
% The sufficiency of $K = 1$ is evaluated by estimating the normalized count-depth relationships. This is done
% by splitting genes into 10 groups based on their non-zero unnormalized median expression (equally sized groups)
% and estimating the mode for each group. If all 10 modes are within .1 of zero, then $K = 1$ is sufficient. If any mode
% is large than .1 or less than -.1, SCnorm will attempt to normalize assuming $K = 2$ and repeat the group normalizations
% and evaluation. This continues until all modes are within .1 of zero. (Simulation studies have shown .1
% generally works well and is default, but may be adjusted using the
% parameter \Robject{Thresh} in the \Rfunction{SCnorm} function).
% If \Robject{PrintProgressPlots = TRUE} is specified, the progress plots of SCnorm
% will be created for each value of $K$ tried (default is \Rcode{FALSE}).
% 
% Normalized data can be accessed using the \Rclass{results()} function.
% 
% <<eval=TRUE>>=
% Conditions = rep(c(1), each= 90)
% pdf("MyNormalizedData_k_evaluation.pdf")
% par(mfrow=c(2,2))
% DataNorm <- SCnorm(Data = ExampleSimSCData,
%                    Conditions = Conditions,
%                    PrintProgressPlots = TRUE,
%                    FilterCellNum = 10, K = 1,
%                    NCores=3, reportSF = TRUE)
% dev.off()
% DataNorm
% NormalizedData <- SingleCellExperiment::normcounts(DataNorm)
% NormalizedData[1:5,1:5]
% 
% @
% 
% The list of genes not normalized according to the filter specifications may be obtained by:
% <<eval=TRUE>>=
% GenesNotNormalized <- results(DataNorm, type="GenesFilteredOut")
% str(GenesNotNormalized)
% @
% 
% If scale factors should be returned, then they can be accessed using:
% <<eval=TRUE>>=
% ScaleFactors <- results(DataNorm, type="ScaleFactors")
% str(ScaleFactors)
% @
% 
% \subsection{Evaluate choice of \textit{K}}
% \label{sec:NormalizationK}
% 
% SCnorm first fits the model for $K = 1$, and sequentially increases K until a
% satisfactory stopping point is reached. In the example run above, $K = 4$ is chosen,
% once all 10 slope densities have an absolute slope mode $< .1$.
% 
% We can also use the function \Rfunction{plotCountDepth} to make the final plot of the
% normalized count-depth relationship:
% <<eval=TRUE, fig.height=5, fig.width=7, fig.align='center', out.width='0.4\\textwidth'>>=
% 
% countDeptEst.SCNORM <- plotCountDepth(Data = ExampleSimSCData, 
%                                       NormalizedData = NormalizedData,
%                                       Conditions = Conditions,
%                                       FilterCellProportion = .1, NCores=3)
% 
% str(countDeptEst.SCNORM)
% head(countDeptEst.SCNORM[[1]])
% @
% 
% \section{SCnorm: Multiple Conditions}
% When more than one condition is present SCnorm will first normalize
% each condition independently then apply a scaling procedure between the conditions.
% In this step the assumption is that most genes are not differentially expressed (DE) between
% conditions and that any systematic differences in expression across the
% majority of genes is due to technical biases and should be removed.
% 
% Zeros are not included in the estimations of scaling factors across conditions by default,
% this will make the means across conditions equal when zeros are not considered (which can
% then used for downstream differential expression analyses with \Biocpkg{MAST}, for example).
% However, if the proportion of zeros is very unequal between conditions and the user plans to
% use a downstream tool which includes zeros in the model (such as \Biocpkg{DESeq} or \Biocpkg{edgeR}),
% then the scaling should be estimated with zeros included in this calculation by using
% the \Rcode{useZerosToScale=TRUE} option:
% 
% <<eval=FALSE>>=
% Conditions = rep(c(1, 2), each= 90)
% DataNorm <- SCnorm(Data = MultiCondData,
%                   Conditions = Conditions,
%                   PrintProgressPlots = TRUE,
%                   FilterCellNum = 10,
%                   NCores=3,
%                   useZerosToScale=TRUE)
% @
% 
% Generally the definition of condition will be obvious given the experimental
% setup. If the data are very heterogenous within an experimental setup it may be
% beneficial to first cluster more similar cells into groups and define these as
% conditions in SCnorm.
% 
% 
% For very large datasets, it might be more reasonable to 
% normalize each condition separately. To do this, 
% first normalize each Condition separate:
% <<eval=FALSE>>=
% 
% DataNorm1 <- SCnorm(Data = ExampleSimSCData[,1:45], 
%                    Conditions = rep("1", 45),
%                    PrintProgressPlots = TRUE,
%                    FilterCellNum = 10,
%                    NCores=3, reportSF = TRUE)
%                    
%                    
% DataNorm2 <- SCnorm(Data = ExampleSimSCData[,46:90], 
%                   Conditions = rep("2", 45),
%                   PrintProgressPlots = TRUE,
%                   FilterCellNum = 10, 
%                   NCores=3, reportSF = TRUE)
% @
% 
% Then the final scaling step can be performed as follows:
% 
% <<eval=FALSE>>=
% normalizedDataSet1 <- results(DataNorm1, type = "NormalizedData")
% normalizedDataSet2 <- results(DataNorm2, type = "NormalizedData")
% 
% NormList <- list(list(NormData = normalizedDataSet1),
%                     list(NormData = normalizedDataSet2))
% OrigData <- ExampleSimSCData
% Genes <- rownames(ExampleSimSCData)
% useSpikes = FALSE
% useZerosToScale = FALSE
% 
% DataNorm <- scaleNormMultCont(NormData = NormList, 
%                   OrigData = OrigData, 
%                   Genes = Genes, useSpikes = useSpikes, 
%                   useZerosToScale = useZerosToScale)
%                   
%                   
% str(DataNorm$ScaledData)
% myNormalizedData <- DataNorm$ScaledData
% @
% 
% 
% \section{SCnorm: UMI data}
% 
% If the single-cell expression matrix is very sparse (having a lot of zeros)
% then SCnorm may not be appropriate. Currently, SCnorm will issue a
% warning if at least one cell/sample has less than 100 non-zero values or
% total counts below 10,000. SCnorm will fail if the filtering criteria or
% quality of data leaves less then 100 genes for normalization.
% 
% If the data have -many- tied count values consider
% setting the option \Rcode{ditherCounts=TRUE} (default is \Rcode{FALSE}) which will help
% in estimating the count-depth relationships. This
% introduces some randomness but results will not change if the
% command is rerun.
% 
% It is highly recommended to check the
% count-depth relationship before and after normalization. In some cases,
% it might be desired to adjust the threshold used to decide K. Lowering the threshold
% may improve results for some datasets (default \Rcode{Thresh = .1}).
% 
% For larger datasets, it may also be desired to increase the speed.
% One way to do this is to change the parameter \Robject{PropToUse}. \Robject{PropToUse}
% controls the proportion of
% genes nearest to the the overall group mode to use for the group fitting(default is \Rcode{.25}).
% 
% 
% <<eval=FALSE>>=
% checkCountDepth(Data = umiData, Conditions = Conditions,
%                 FilterCellProportion = .1, FilterExpression = 2)
% 
% DataNorm <- SCnorm(Data = umiData, Conditions= Conditions,
%                   PrintProgressPlots = TRUE,
%                   FilterCellNum = 10,
%                   PropToUse = .1,
%                   Thresh = .1,
%                   ditherCounts = TRUE)
% @
% 
% 
% \section{Spike-ins}
% 
% SCnorm does not require spike-ins, however if high quality spike-ins
% are available then they may be use to perform the between condition scaling step. If
% \Rcode{useSpikes=TRUE} then only the spike-ins will be used to estimate the scaling
% factors. If the spike-ins do not span the full range of expression, SCnorm will
% issue a warning and will use all genes to scale. SCnorm also assumes the spikes-ins are named with the prefix "ERCC-".
% 
% <<eval=FALSE>>=
% 
% ExampleSimSCData.SCE <- SingleCellExperiment::SingleCellExperiment(assays = list('counts' = ExampleSimSCData))
% 
% ## Assuming the spikes are ERCC, otherwise specify which features are spike-ins manuallly using the splitAltExps function.
% myspikes <- grepl("^ERCC-", rownames(ExampleSimSCData.SCE))
% ExampleSimSCData.SCE <- SingleCellExperiment::splitAltExps(ExampleSimSCData.SCE, ifelse(myspikes, "ERCC", "gene"))
% 
% DataNorm <- SCnorm(Data = ExampleSimSCData.SCE, Conditions = Conditions,
%                    PrintProgressPlots = TRUE,
%                    FilterCellNum = 10, useSpikes=TRUE)
% @
% 
% \section{Within-sample normalization}
% 
% SCnorm allows correction of gene-specific features prior to the between-sample
% normalization. We implement the regression based procedure from
% 
% Risso et al., 2011 in \Biocpkg{EDASeq}. To use this feature you must set withinSample equal to a
% vector of gene-specific features, one per gene. This could be anything,
% but is often GC-content or gene length.
% 
% A function to evaluate the extent of bias in expression related to the gene-specific feature is \Rcode{plotWithinFactor()}.
% Genes are split into 4 (NumExpressionGroups = 4 is default) equally sized groups
% based on the gene-specific factor provided. For each cell the median expression of genes
% in each group isestimated and plot. The function \Rcode{plotWithinFactor}
% returns the information used for plotting as a matrix of the expression medians for each group for all cells.
% 
% In this example I generate a random vector of gene specific features, this may be the proportion of GC content for example.
% Each cell receives a different color
% 
% <<eval=TRUE, fig.height=5, fig.width=10, fig.align='center', out.width='0.8\\textwidth'>>=
% #Colors each sample:
% exampleGC <- runif(dim(ExampleSimSCData)[1], 0, 1)
% names(exampleGC) <- rownames(ExampleSimSCData)
% withinFactorMatrix <- plotWithinFactor(Data = ExampleSimSCData, withinSample = exampleGC)
% 
% head(withinFactorMatrix)
% @
% 
% Specifying a Conditions vector will color each cell according to its associated condition:
% <<eval=TRUE, fig.height=3, fig.width=4, fig.align='center'>>=
% #Colors samples by Condition:
% Conditions <- rep(c(1,2), each=45)
% withinFactorMatrix <- plotWithinFactor(ExampleSimSCData, withinSample = exampleGC,
%                                        Conditions=Conditions)
% 
% head(withinFactorMatrix)
% @
% 
% <<eval=FALSE>>==
% # To run correction use:
% DataNorm <- SCnorm(ExampleSimSCData, Conditions,
%                    PrintProgressPlots = TRUE,
%                    FilterCellNum = 10, withinSample = exampleGC)
% 
% @
% 
% For additional evaluation on whether to correct for these features or other options for
% correction, see: Risso, D., Schwartz, K., Sherlock, G. \& Dudoit, S. GC-content
% normalization for RNA-Seq data. BMC Bioinformatics 12, 480 (2011).
% 
% 
% 
% \section{Session info}
% Here is the output of sessionInfo on the system on which this document was
% compiled:
% <<eval=TRUE>>=
%   print(sessionInfo())
% @
% 
% \section{Frequently Asked Questions}
% 
% {\bf \large Does SCnorm work for all datasets?}
% 
% No, while SCnorm has proven useful in normalizing many single-cell RNA-seq datasets, it will not be appropriate in every setting. It is important to evaluate normalized data for selected genes as well as overall (e.g. using plotCountDepth).
% 
% 
% {\bf \large Can I use SCnorm on your 10X (or very sparse) dataset?}
% 
% SCnorm is not intended for datasets with more than $\sim$80\% zero counts, often K will not converge in these situations.
% Setting the FilterExpression parameter to 1 or 2 may help, but is not a guarantee.
% It may also be helpful to use the ditherCounts = TRUE parameter for sparse UMI based data which may contain numerous tied counts (counts of 1 and 2 for example).
% 
% 
% {\bf \large Can/Should I use SCnorm on my bulk RNA-seq data?}
% 
% SCnorm can normalize bulk data. However, SCnorm should not be used to normalize data containing less
% then 10 cells/samples. Consideration must also be given to what downstream analysis methods will be used.
% Methods such as DESeq2 or edgeR typically expect a matrix of scaling factors to be supplied, these
% can to be obtained in SCnorm by setting \Rcode{reportSF=TRUE} in the SCnorm function.
% 
% The count-depth relationship of both unnormalized and normalized bulk RNA-seq data
% can be evaluated using the \Rfunction{plotCountDepth} function.
% 
% {\bf \large How can I speedup SCnorm if I have thousands of cells?}
% 
% Utilizing multiple cores via the \Rfunction{NCores} parameter is the most effective way.
% 
% Splitting the data into smaller clusters may also improve speed.
% Here is one example on how to cluster and the appropriate way to run \Rcode{SCnorm}:
% (This method of clustering single-cells was originally suggested by Lun et al., 2016 in \Biocpkg{scran}.)
% 
% <<eval=FALSE>>=
% library(dynamicTreeCut)
% distM <- as.dist( 1 - cor(BigData, method = 'spearman'))
% htree <- hclust(distM, method='ward.D')
% clusters <- factor(unname(cutreeDynamic(htree, minClusterSize = 50,
%                     method="tree", respectSmallClusters = FALSE)))
% names(clusters) <- colnames(BigData)
% Conditions = clusters
% 
% DataNorm <- SCnorm(Data = BigData,
%                   Conditions = Conditions,
%                   PrintProgressPlots = TRUE
%                   FilterCellNum = 10,
%                   NCores=3, useZerosToScale=TRUE)
% @
% 
% {\bf \large When should I set the parameter \Rcode{useZerosToScale=TRUE} ?}
% 
% This parameter is only used when specifying multiple conditions in SCnorm. The \Robject{useZerosToScale} parameter
% will determine whether zeros are used in estimating the condition based scaling factors.
% 
% Use the default, \Rcode{useZerosToScale=FALSE}, when planning to use single-cell specific methods such as MAST
% that separately model continuous and discrete measurements.
% 
% If you plan on using an analysis method that explicitly uses zeros in the model like DESeq2,
% then \Rcode{useZerosToScale=TRUE} should be used.
% 
% {\bf \large SCnorm was unable to converge.}
% 
% SCnorm will fail if $K$ exceeds 25 groups. It is unlikely that such a large number of groups is appropriate. This
% might happen if a large proportion of your genes have estimated count-depth relationships very close to zero.
% Typically this error can be resolved by removing genes with very small counts from
% the normalization using the parameter \Robject{FilterExpression}. Genes having non-zero median expression
% less than \Robject{FilterExpression} will
% not be scaled during normalization. An appropriate value for \Robject{FilterExpression} will be data dependent and decided based on
% exploring the data. One way that might help decide is to see if there are any natural cutoffs in the non-zero medians:
% 
% <<eval=FALSE>>=
% MedExpr <- apply(Data, 1, function(c) median(c[c != 0]))
% plot(density(log(MedExpr), na.rm=T))
% abline(v=log(c(1,2,3,4,5)))
% # might set FilterExpression equal to one of these values.
% @
% 
% \vspace{1cm}
% %\bibliographystyle{natbib}
% 
% 
% 

\end{document}
